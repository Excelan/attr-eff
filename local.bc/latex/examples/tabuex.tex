\input{../templates/DMS}
\pagenumbering{gobble}

\fancypagestyle{maintext}
{
\pagenumbering{gobble}
   \fancyhf{}
  \renewcommand{\headrulewidth}{0pt} % убрать линию в верхнем колонтитуле
  \renewcommand{\footrulewidth}{0pt} % добавить линию
   \fancyhead{}
   \fancyhead[R]{
   \begin{tabularx}{\textwidth}{  X  X  }
   \\
   \end{tabularx}
  }
   \fancyhead[C]{\thepage}
   \fancyfoot{}
\fancyfoot[C]{\begin{tabular}{rr}
                \noindent\rule{5cm}{0.5pt} \varSignature{}   & \noindent\rule{4cm}{0.5pt} \varDirector{}
              \end{tabular}%
}

}

\newcounter{example}[section]
\newenvironment{example}[1][]{\refstepcounter{example}\par\medskip
\\
\\
   \textbf{Приложение~\theexample #1 \\к Договору  №  \varId{}  \\ от \changedate{\varDate}   \\  } \rmfamily}{\medskip}

	 \fancypagestyle{dodatki}
    {
    	\fancyhead{}
     \fancyhf{}
       \fancyhead[R]{
      \begin{flushright}
    	\begin{example}
    	 .
    	\end{example}
      \end{flushright}
      }
       \renewcommand{\headsep}{40pt}
       \fancyfoot{}
       \fancyfoot[C]{\begin{tabular}{rr}
                    \noindent\rule{5cm}{0.5pt} В.В. Павлова & \noindent\rule{4cm}{0.5pt} И.О. Фамилия
                  \end{tabular}%
    }
    }


%\setlength{\headheight}{60.0pt}
%\setlength{\footskip}{70.99991pt}


\newcommand{\ValueDocumentNumber}{123}
\newcommand{\ValuePlace}{с. Б. Александровка}
\newcommand{\ValueDate}{1 февряля 2016 года}

% http://tex.stackexchange.com/questions/85178/increase-height-of-fancyhdr-header
% !!!! http://tex.stackexchange.com/questions/117804/how-to-determine-head-height-automatically
\pagestyle{fancy}
	\rhead{
\begin{tabularx}{\textwidth}{X  X  X  X  }
LOGO & Класс документа:    & Id  документа   & Название документа:     \\
 &RegulationSOP        & 1        & RegulationSOP513         \\
&Тип документа:    & № Версии документа: & Страница в документе:  \\
&СОП    &  503 & \thepage  \\
\end{tabularx}
}

\fancyfoot[C]{
\begin{tabularx}{\textwidth}{X X X}
\tiny Разработал :    & Согласовал : & Утвердил:      \\
\tiny Иванов  Ассистент отдела  21.01.2016 9:30        &  Ларцев  Заместитель менеджера    21.01.2016 11:30 & Петров Менеджер  21.01.2016 15:30   \\
\hline
\tiny Дата вступления в силу :   & Дата последнего пересмотра: & Период действия:    \\
\tiny 25.01.2015 & 22.01.2015 & 1 year       \\
\end{tabularx}}

\renewcommand{\headrulewidth}{1pt}
\renewcommand{\footrulewidth}{1pt}



\begin{document}

%\pagestyle{maintext}

\begin{center}
{\large \textbf{\textsc{договор аренды нежелых помещений № \ValueDocumentNumber}}}
\end{center}

\begin{tabularx}{\textwidth}{ X rX }
\ValuePlace & \ValueDate \\
\end{tabularx}

Общество с ограниченной ответственностью ХФК «\textbf{Биокон}»,
являющееся плательщиком налога на прибыль на общих условиях, далее -

Арендодатель, в лице директора Павловой В. В., которая действует на
основании Устава, с одной стороны, и OOO «XXX», Далее - Арендатор, в
лице ФИО, действующего на основании Устава, с другой стороны, а при
упоминании вместе - Стороны, заключили Этот Договор о нижеследующем:

{\small \begin{center}
\tabulinesep=2mm
\begin{longtabu} to \textwidth { | X[0.5,c,p] | X[0.5,l,p] | X[1,l,p] | }  % {|X|X|X|}
\hline
\textbf{Type 1 of movement in the overlap zone} & \textbf{State of the object in camera 1} & \textbf{State of the object in camera 2}  \\ \hline
\endfirsthead
%\endhead
\hline
\endfoot
\caption{Table title 2.}
\endlastfoot

11111Object moving from camera 1 to camera 2 & Object will eventually be in the \emph{exiting} state and the be \emph{deleted} & Object will be recognised as a \emph{new} object, then be come \emph{to be classified} and eventually \emph{classified} \\ \hline

Object moving camera 2 & Object will in the \emph{exiting} state and the be \emph{deleted} & Object will be recognised as a \emph{new} object, then become \emph{to be classified} and eventually \emph{classified} \\ \hline

Object moving from camera 1 to camera 2 & Object will eventually be in the \emph{exiting} state and the be \emph{deleted} & Object will be recognised as a \emph{new} object, eventually \emph{classified} \\ \hline

Object moving from camera 1 to camera 2 & Object will eventually be in the \emph{exiting} state and the be \emph{deleted} & Object will be recognised as a \emph{new} object, then become \emph{to be classified} and eventually \emph{classified} \\ \hline

Object moving from camera 1 to camera 2 & Object will eventually be in the \emph{exiting} state and the be \emph{deleted} & Object will be recognised as a \emph{new} object, then be come \emph{to be classified} and eventually \emph{classified} \\ \hline

Object moving camera 2 & Object will in the \emph{exiting} state and the be \emph{deleted} & Object will be recognised as a \emph{new} object, then become \emph{to be classified} and eventually \emph{classified} \\ \hline

Object moving from camera 1 to camera 2 & Object will eventually be in the \emph{exiting} state and the be \emph{deleted} & Object will be recognised as a \emph{new} object, eventually \emph{classified} \\ \hline

Object moving from camera 1 to camera 2 & Object will eventually be in the \emph{exiting} state and the be \emph{deleted} & Object will be recognised as a \emph{new} object, then become \emph{to be classified} and eventually \emph{classified} \\ \hline

Object moving from camera 1 to camera 2 & Object will eventually be in the \emph{exiting} state and the be \emph{deleted} & Object will be recognised as a \emph{new} object, then be come \emph{to be classified} and eventually \emph{classified} \\ \hline

Object moving camera 2 & Object will in the \emph{exiting} state and the be \emph{deleted} & Object will be recognised as a \emph{new} object, then become \emph{to be classified} and eventually \emph{classified} \\ \hline

Object moving from camera 1 to camera 2 & Object will eventually be in the \emph{exiting} state and the be \emph{deleted} & Object will be recognised as a \emph{new} object, eventually \emph{classified} \\ \hline

Object moving from camera 1 to camera 2 & Object will eventually be in the \emph{exiting} state and the be \emph{deleted} & Object will be recognised as a \emph{new} object, then become \emph{to be classified} and eventually \emph{classified} \\ \hline

Object moving from camera 1 to camera 2 & Object will eventually be in the \emph{exiting} state and the be \emph{deleted} & Object will be recognised as a \emph{new} object, then be come \emph{to be classified} and eventually \emph{classified} \\ \hline

Object moving camera 2 & Object will in the \emph{exiting} state and the be \emph{deleted} & Object will be recognised as a \emph{new} object, then become \emph{to be classified} and eventually \emph{classified} \\ \hline

Object moving from camera 1 to camera 2 & Object will eventually be in the \emph{exiting} state and the be \emph{deleted} & Object will be recognised as a \emph{new} object, eventually \emph{classified} \\ \hline

Object moving from camera 1 to camera 2 & Object will eventually be in the \emph{exiting} state and the be \emph{deleted} & Object will be recognised as a \emph{new} object, then become \emph{to be classified} and eventually \emph{classified} \\ \hline

\end{longtabu}
\end{center}
}




\section{\large \textbf{\textsc{предмет договора}}}
% \thispagestyle{empty}
\begin{enumerate}
\tightlist
\item
  \textbf{Арендодатель} передает, а Арендатор принимает в аренду нежилое
  помещение (далее - арендуемое помещение) расположеное по адресу:
  Киевская обл., Бориспольский р-н., с.Большая Александровка, ул.
  Бориспольская, 9, общей площадью ХХХ кв. м. (План помещения приведен в
  Приложении № 2 к настоящему Договору).
\item
  Арендуемое помещение принадлежит Арендодателю на правах собственности
  в соответствии со Свидетельством о праве собственности на недвижимое
  имущество серия CАК 136135 от 01.07.2013 г.. (Копия Свидетельства
  прилагается).
\item
  Помещение передается в пользование исключительно с целью его
  использования под офис.
\end{enumerate}


\section{\large \textbf{\textsc{порядок передачи объекта в аренду}}}
No




\end{document}
